\section{Evaluation}\label{sec:evaluation}

In this section, we present several applications of imputing objective functions. Furthermore, for each case, we first use mathematical modelling to get a convex optimization problem, then we design or apply our approaches and algorithms before implementation. With processing of experimental data and diagrams, we lead a discussion on the performance as well as future work.

\subsection{Curve Fitting}

Curve fitting (\haoyuan{ref}) has widespread applications in data processing among various areas, including big data, machine learning, simulation, noise reduction and so on (\haoyuan{ref}). Interpolation is known as an effective approach for curve fitting. However in this model, we mainly concentrate on a situation of implicit curve fitting. That is to say, we derive an approximate form of the original function by observing the extreme-value points.

We start from a simple example. Suppose the primal function is $f(x)=ax^2+bx$. The values $a$ and $b$ cannot be observed, and hence can only be derived from a set of sample points. As we discussed in the last section, the objective function can be any multiple of $f(x)$. To ensure a unique solution, we model this problem as below.
\begin{align}
\begin{split}\label{quad}
\textrm{minimize }\hspace{.2in}&px+f(x)\\
\textrm{subject to:}\hspace{.2in}&x\ge0
\end{split}
\end{align}

Note that we require $a>0$ so that~(\ref{quad}) is a parameterized convex optimization problem, and we only care about nonnegative solutions. The residues of KKT conditions are
\begin{align*}
r_1 &= -\lambda x\\
r_2 &= p+\nabla f(x)-\lambda
\end{align*}

By~(\ref{p3}), suppose a list of $S$ sample points $(x^*,p)$ are observed from the primal problem~(\ref{quad}). If we assume another quadratic function $\tilde{f}(x)=mx^2+nx$ to fit $f(x)$, we derive the values of $m,n$ by
\begin{align*}
\begin{split}
\textrm{minimize }\hspace{.2in}&\sum_{s=1}^S\left[(2mx_s^*+n+p_s-\lambda_s)^2+(-\lambda_s x_s^*)^2\right]\\
\textrm{subject to:}\hspace{.2in}&\lambda_s\ge0,\hspace{.1in}s=1,\cdots,S
\end{split}
\end{align*}

Another approach for this problem: since $px+\tilde{f}(x)$ reaches the extreme value at $x=-\frac{n+p}{2m}$, we can easily figure out $m,n$ with those sample points by interpolation. But we are not to implement this approach, as we focus more on the performance of the parameterized programming.

Next we conducted the experiment using Matlab 2014a. We randomly select $a=1$, $b=-10$ for the primal problem. 100 sample points are acquired by setting $p\in[2,8]$ with a uniform distribution and solving the primal problem.

\subsection{Decision Making in Game Theory}
